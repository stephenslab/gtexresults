\subsection{Methods Overview: Approach} 
We aim to learning about patterns of sharing across tissues within a SNP and among SNPs, which join to help us better understand the global and snp-specific patterns of effects of genetics on gene expression. 
This allows us to make comparisons among tissues in which the QTL is called active, and among gene-snp pairs with a similar degree of activity in a given tissue. 
Thus as an additional level of combining information, we assume that each eQTL may follow a particular pattern of activity characterized by its effects across tissues. Within these groups, the tissues exhibit characteristic patterns of sharing, which can be captured by considering the covariance structure of the genetic effects among tissues. \insert{figure}This lends itself to a mixture model, in which  we assume all the gene-snp pairs arise from a mixture of a finite number of multivariate normal (MVN) distribution, each characterised by the covariance matrix from which the vector of effects is though to arise. For each of $J$ gene-snp pairs, we observe an R dimensional vector of standardized effect sizes $\bm\hat{b}$ and their standard error and assume that these effects descend from some true effect size $\bm{b}$. 



 \begin{equation}
  \bm{b_{j}} | \bm{\pi},\bf{U} \sim \sum_{k,l} \pi_{k,l} \;{\it N}_R(\bm{0}, \omega_l U_{k})
\end{equation}

Here, the covariance matrix $U_{k}$ captures the particular patterns of sharing - variation in effect sizes within and between tissues, while $\omega_{l}$ determines the scale of each pattern - the magnitude of the effect size. Thus we recognize that while two eQTL may obey a similar pattern or shape, the absolute scale may vary. For example, two eQTL may both have strong correlation between tissues 1 and 2 with consistently larger effects in 2, but the absolute size of the effects may vary between SNPs.
 
Previous work from our lab considered the idea of configuration - i.e., that a tissue was simply `active' or `inactive' in a particular tissues - and thus for R tissues, there were 2^{R} possible configurations. which becomes computationally infeasible as R grows.
Furthermore, this considered only the idea that the variance in effect sizes between two tissues was the same across tissues thought to be active and the covariances were also the same among tissues though to be active in a given `configuration',  and thus failed to incorporate the much richer covariance structure between tissues.
As a critical innovation on our previous method $(\cite{flutre_statistical_2013,wen_bayesian_2014})$ these matrices contain distinct diagonal and off-diagonal elements which reflect data-specific patterns of variation within and covariance between subgroups (tissues). This captures the variation in effect sizes within and between subgroups better than restricting effects to simply 'shared' or 'unshared' between subgroups. 
 
Because we can't know the 'true covariance matrix' for each gene-snp pair, we aim to assemble a list which sufficiently captures the various patterns, and then 'learn' the relative proportions of each pattern of sharing from the data. One can now model each vector of effect sizes $\bm{b}$ each as arising from a mixture that captures all the covariance patterns.

The primary novelty of this approach is {\it to estimate this multivariate posterior distribution on the effect size in a data-sensitive way} - i.e., using the mixture model to capture information about the covariance structure among subgroups (here, tissues). Thus we might identify a situation in which it is common to have large effects in some tissues and not others, and thus if a gene-snp pair demonstrates a small effect in one of the 'off issues', we might be inclined to conclude that it is indeed a member of this particular class and shrink the small effect in this tissue accordingly. However, if we see the same small effect in a setting in which 'similar tissues' have large effects, we might 'shrink' this effect size less, due to our high prior belief in the SNP's effectiveness garnered from adjacent tissues. Thus we deem this method 'adaptive Shrinkage' because the appropriate amount of shrinkage is learned from the data. 

Because our prior belief in consistency is strong, we identify many more ?significant associations? in setting where perhaps the observed univariate statistic in one tissue is small but otherwise large in additional tissues, nudging these effects towards something more consistent. This is in contrast to a univariate shrinkage approach, in which all effects of the same size would be 'shrunk' equivalently, due to lack of information garnered from adjacent tissues. 

In fact, Shrinkage towards 0 of small effects is a result, not a necessity - since the majority of the prior weight is on small $\omega$ components which emphasize components with small prior variance of the effect size $\textbm{b}$, many of the modest z statistics will be smoothed or shrunk towards the prior mean, 0. 
An additional novelty is that in learning something about the effect size in each tissue for a given gene-snp pair, we can make statements about the degree of heterogeneity - that is the proportion of the time we expect a SNP to have effects of different sign. We will be confident in our ability to identify the direction of the effect for A SNP with a large effect and relatvilely high precision, and thus we can use an estimate of the posterior mean in each component and the proportion to quantify the distribution of gene SNP pairs who have effects of opposite direction (or lack convincing evidence of effects in a consistent direction across tissues).

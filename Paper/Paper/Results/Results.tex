\section{Results}
\subsection{Demonstrating Features of the Method}

To get a sense of the accuracy of our novel approach to estimating multivariate effects, we simulated two types of data. 

In the first set, in which we expect our method to be superior to both univariate methods and methods in which the configuration approach is utilized, we simulate 50,000 gene-snp pairs, with only 400 representing true signal. This represents roughly 500 genes with 100 snps in cis,  $80\%$ of which contain one active QTL. Thus naturally, if the gene contains such a QTL, it is the same QTL among all tissues in which the tissue is active. This puts a dual burden on both features of the method: The small number of true associations tests whethere the method accurately allowing pushes small observed effects toward zero while hoping to preserve the true signal when it exists. Furthermore, the  multivariate nature of these effects when they exists tests the ability of the method to accurately infer patterns of sharing from the dataset. These true effects are thus simulated from the 'learned' covariance matrices representing $U_{k}$ 2-9, and thus aim to emulate the patterns of sharing present in real biological data. We compare with univariate 'shrinkage' method ASH (Stephens et al, unpublished) as well as the eqtlBMA-lite (Flutre et al, 2013) which uses the singleton and fully consistent (i.e., active in only one tissue, or active with the same effect size in all tissues) configurations to estimate these effects jointly. We call this the 'sharing' (S) scenario. 

One might expect that our method would prove superior only in the setting in which true effects are shared among all tissues, and thus fail in the setting of tissue specificity. Thus, building on the situation above, we add a simulation in which $35\%$ of the true effects are active in only one tissue, according to 5 different patterns of tissue specificity. We call this the 'tissue-specific' (TS) scenario. 

We see that in terms of both power and accuracy, MASH is superior in both setting to both univariate methods, and to joint analysis under the setting of orthogonal configurations.


\begin{equation}
RMSE_{HI} \sqrt(\sum_{jr}(true_{b_{jr}}-estimated_{{b_{jr}})^2))$
\end{equation}
\begin{table}[ht]
\caption{Accuracy Comparison: RMSE}
\centering
\begin{tabular}{c c c c}
\hline\hline
Inference Method & MASH & ASH & eqtlBMA-lite \\ [0.5ex] % inserts table %heading
\hline
RMSE_{S}&0.091&0.28&0.43\\
RMSE_{TS}&0.08& 0.23&0.39 \\%&8649
cor.with.truth_{S}&0.99&0.94&0.84\\
cor.with.truth_{TS}&.99&0.94&0.82\\
\hline
\end{tabular}
\label{table:RMSE}
\end{table}

To demonstrate the ability of MATRIX ASH to powerfully capture these accurately estimated effect sizes, we compare the proportion of true associations called significant at a given signficance threshold among the three methods. Indeed, MASH  proves superior to both methods under each condition (i.e., sharing or tissue-specific). 

\texbf{INSERT POWER VS ACCURACY FIGURE HERE}

In introducing a method to quantify the heterogeneity of effect sizes, we have developed a 'heterogeneity index' which attempts to capture the heterogentity in magnitude among tissues in which the gene-snp pair is active. For each gene-snp pair, we normalize its vector of effects across tissues by the effect which has the maximum absolute value; thus for a fully 'consistent' gene-snp pair in which all the effects are equal in magnitude, the new vector of normalized effects would consist of all ones, and 44 tissues would be greater than 43 tissues would be greater htan $50\%$ of the maximum effect. By contrast, for a tissue-specific gene-snp pair, the vast majority of effects would be small fraction of the maximum effect and thus the number of tissues greater than $50\%$ of the maximum effect would be 1. We will apply these heterogeneity index quantities, here deemed 'HI' to the real data, but first wanted to demonstrate the superiority of MATRIX ASH in estimating these quantities on simulated data. To quantity the ability of each method to accurately ascertain the heterogeneity, we can compute the heterogeneity index of the real data, and the inferred quantities, and use a modified RMSE:

\begin{equation}
RMSE_{HI} \sqrt(\sum(truth_{HI}-estimated_{HI})^2))$
\end{equation}

\begin{table}[ht]
\caption{Accuracy Comparison: RMSE}
\centering
\begin{tabular}{c c c c}
\hline\hline
Inference Method & MASH & ASH & eqtlBMA-lite \\ [0.5ex] % inserts table %heading
\hline
HI_{S}&39.38 &40.87 &39.78 \\
HI_{TS}& 39.98& 40.77&39.51\\
\hline
\end{tabular}
\label{table:HETindex}
\end{table}

\subsection{Adaptive Shrinkage: The Multivariate Approach}

To demonstrate the utility of shrinking effect size estimates jointly, we consider the estimated effect sizes vs their observed input summary statistics using our joint (MASH) and comparing to a univariate shrinkge method (ASH). On simulated data, we can also then plot the estimated effect sizes against the true values,  again comparing among methods. Here, we show the results under the setting of tissue specificity, to analyze the behavior of eQTL of each class. 


\texbf{INSERT TSPEC SIMULATED SCATTERPLOTS HERE}

Comparing estimated Z statistics vs the observe 'raw' input valuesIn both MASH and univariate methods, values with large standard errors will be shrunk more harshly. Furthermore, in this simulated data, where there is an abundance of small effects, tend to  shrink small z statistics towards prior mean at 0 as their likelihoods will be maximized by componenent with small $\omegas$. However, now considering z statistics of the same size (so now accounting for standard erro), not all small values are shrunk to the same extent using MASH, due to the power of joint analysis to consider the effects across tissues in inferring the final vector of effect sizes/ Acknowledging consistency, small effects in one tissue will be augmented in the presence of larger effects present in other tissues, resulting in dramatic power increases. 

Furthermore, when we plot the estimated effects vs the truth and segregate these effects by class, we see that the correlation among the truth and estimated effect sizes is much tighter using our multivariate approach. Similarly, truly null effects are shrunk more tightly, due to the fact that in the presence of consistency, small effects across subgroups will lead us to have a high prior belief that an additional small observed effect in that eQTL is also likely to be close to 0. Importantly, tissue specific SNPs are still captured using our joint approach, demonstrating that have tissue-specific patterns exist in the data, our prior belief will capture this phenomenon and accordingly our posterior estimates will reflect the underlying tissue specific nature of these activities.

Together, these results demonstrate the tremendous power increase of using a multivariate method and the accuracy of estimating patterns of sharing from the data rather than imposing forced configurations which fail to capture the heterogeneity of effect sizes among tissues.


\section{Real Data}

Now, we consider the results of our analysis, when applied to the GTEX data set. After estimating the covariance matrices from the strongest Z statistics in the data, thus demonstrating the strong underlying 'true patterns' of sharing in the data and adding the qualitatively specific configurations, we then inferred the relative frequency of each pattern of sharing and corresponding effect sizes from a large sample of 40,000 gene snp pairs. Here, we report the analysis on the top SNP for each of 16,069 genes, where the 'top' snp is defined as the SNP with the largest effect size in absolute value across tissues. As described above and demonstrated in simulations, in the setting of an abundance of small effects in data set, we  tends to shrink small z statistics towards the prior mean at 0. It should be noted that this is a result specific to a particular data set, and in that sense 'adaptive' - indeed, if small effects were rare and large effects abundant, such shrinkage would not occur. 

But perhaps more importantly, the striking increase in power when compared to univariate methods is noted. There are a total of 44 tissues x 16,069 gene-snp pair associations considered, or 707,036 total tissue-level effect size coefficients. At an $lfsr$ threshold of 0.05, we identify 393,414 significant snp-gene-tissue effects. Using the naive univariate summary statistics and using an FDR threshold of 0.05, we identify only 202,087, meaning that using univariate methods we would only call an effect non-zero in less than a third $(28\%)$ of cases, while using our posterior effect estimation we would say that we can confidently argue the SNP has a non-zero effect for a gene in a particular tissue over half (55\%) of the time. We would expect this increase in power, because small Z statistics in a tissue will be increased for a given gene in the present of larger z statistics in other tissues.

\textbf{SHOW REAL DATA SCATTERPLOT}

\begin{table}[ht]
\caption{Power Comparison}
\centering
\begin{tabular}{c c c c}
\hline\hline
Metric & LFSR_{MASH} & LFSR_{ASH} \\ [0.5ex] % inserts table %heading
\hline
Significant $\bf{b}_{jr}$ $\leq$ 0.05%&202087
&393414 & 91755&401552\\
$\bf{b}_{jr}$ significant in other not in MASH %&8649
&NA&1447 \\
$\bf{b}_{jr}$ significant in MASH not in other %&199976
&NA&303106 \\[1ex]
\hline
\end{tabular}
\label{table:power}
\caption{\textbf{Power} Restricting our analysis to thresholding by local false sign rates, we can quantify the number of associations we identify at a given local false sign rate threshold using the original summary statistics and posterior means computed using multivariate Matrix Ash and Univariate Ash. We can see that Matrix Ash calls nearly twice and 4 times as many associations significant, and misses very few associations identified in other methods. These are rare examples in which modestly large effects in one tissues are `nudged' towards small effects as observed in the additional tissues for that SNP.}
\end{table}

While the number of associations catpure is slightly greater using the BMA lite approach, we note that the likelihood of the data set under this model is much much worse (1298672 vs -1267997.5). Indeed, eQTLBMA would put the vast majority of the prior weight on the fully 'consistent' configuration, as SNPs demonstrating activity acorss all tissues, regardless of how heterogeneous among subgroups, are forced into this configuration.  

\textbf{SHOW MASH PRIOR WEIGHT VS BMA}

Considering a two tissue example, a configuration type approach recongizes only patterns constrained to lie along the x and y axis or along the x-y line. Matrix ash allows for patterns which show consistenty larger effects in one tissue over another, with varying amounts of correlation among tissues. In these example from real data, we can see that while SNPS of the green, blue and yellow class appear consistently active in both tissues plotted (Brain and Muscle), the blue and yellow effects have consistently larger effects in Muslce then brain, while SNPs of the green class show the reverse. Similarly, SNPS of the yellow close show only loose correlation among the pair of tissues, while SNPs of the green class show strong prediction of activity in one from activity in the other.  Similarly, we can see that SNPS of the pink class tend to show tisu-specificity in Whole Blood relative to testis, while SNPs of the black class show tissue specificity in testis relative to whole blood.

\textbf{SHOW COLORED SCATTERPLOT}

\subsection{ A qualitative description of heterogeneity in the GTEX data}

Indeed, from the prior weight assigned to the 'learned matrices' above coupled with the simulation results in the previous sections, we can see that Matrix ASH is able to accurately parse shared configurations. Focusing on the two predominant patterns, we see that  Here we examine several examples with a high posterior probability of arising from this particular component while the learned matrix $U_{k}$ = 3 seems to capture gene-snp pairs with large, correlated effects in brain, matrix $U_{k}$ = 2 captures SNPs with small effects in brain and larger effects in thyroid and transformed cell-types (e.g., fibroblasts, lymphocytes). The lower which rank high in importance (e.g., $U_{k} 5 and $U_{k}$ = 9) show somewhat tissue specific (i.e., high prior variance in only one tissue-type) effects in testes and whole blood, consistent with our conclusions that whole blood and testes indeed demonstrate an abundance of tissue-specific gene-snp pairs.


\subsection{Examples of strong loading on Uk3, Uk9 and eQTL-BMA lite}


\textbf{Uk3: Captures correlation in sign, quantitative heterogeneity in magnitude along diagonal emphasizes the utility of continuous approach}
In this particular example, strong effects in brain matching an underlying pattern of shared effects among brain tissue is well-captured by the data and thus allows this gene-snp pair to find its true match. brain effect sizes thus borrow strength from one another, and the posterior estimates tend to nudge the brains towards a consistent, shared effect. Similarly, an overall tendency towards consistency tends to 'flip' erratic off directions towards the prevailing positive direction.

%Flip 'erratic' off directions, but maintain ability to recognize when we lack confidence in sign of the effect
\textbf{Uk 9: Quantitative specificity in magnitude in testes/Whole Blood}
In this example, though this paritculr pattern captures correlation in sign among all tissues, the quantitative heterogeneity is reflected in the intnesity of the badning along the diagonal, and thus intorduces the idea of quantitative specificity - e.g., that a SNP can be modestly 'active' in all tissues though to dramatically different degrees. here, though this matrix was learned (and not forced, as in eQTLBMAlite) from the data, the pattern of quantitative tissue specificity in testes and whole blood is evident. Again, erratic, off-directions are flipped in sign.

Lastly, the inclusion of the eQTLBMA lite configurations (in which the SNP has a non-zero effect in only one tissue) coupled with the learned patterns of tissue specificit evident in matrices $U_{k}5-9$ serve to allow the preservation of qualitatively specific effecs. Here, we show a gene-snp pair demonstrating high loading on one of the eqtlbma lite configuration matrices - indeed, we reject the significance of the effect size estimates in all tissues but testes, a pattern consistent with the presence of tissue-specificity described below.


\subsection{Tissue Specificity}

One of the criticisims of a joint approach might be its loss of tissue-specificity. That is, by considering effects across subgroups in estimating the effect size, one might lose sight of tissue specific activity when it exists. Here, we demonstrate our ability to recongize such specificity both quantitatively (through learned patterns of sharing which specifiy consistently larger effects in one tissue over others) and qualitatively (through forced prior effect size mass on 0).  For each tissue, we can ask how many gene-snp pairs meet a given tsignficiance threshold in that tissue alone. 
\textbf{NUMBER OF QTL PER TISSUE PLOT}

Furthermore, tissue speicific eQTL demonstrate the smoothing feature of this joint shrinkage approach: for gene SNP pairs which demonstrate strong effects in only one tissue, the weaker errativ tissue are shrunk towards the prior mean at 0, resulting in a tissue specific smoothing.

\textbf{TISSUESPECIFIC SMOOTHING PLOT}

\subsection{Quantifying Heterogeneity}


Armed with a vector of effect size estimates across 44 tissues, we can move beyond asking in how many tissues is a given gene-snp pair signficant, and ask about the relationship in effect size and direction among tissues in which the gene-snp pair is active. From a biological standpoint, we might consider think that effects of a different sign are rare. Similar to the heterogeniety index described in the simulation framework above which attempted to describe heterogeniety in magnitude, we can plot the number of tissues in which the sign is differ than the effect with maximum absolute value. Considering this results with and without the inclusion of the brain tissues, which appear to behave as a strongly correlated group, we observe several phenomenon. The majority of gene-snp pairs are consistent in sign (indeed, only about 20\% of genes show two signficiant effects of a different sign when including brain, and even fewer (14.8\%) when excluding brains) and removing brains from our analysis tends to push us towards consistency, suggesting that brain appears to behave as a large tissue-specific entity.

\textbf{Sign Heterogeneity Hist}

Furthermore, we can now quantify the heterogientiy index in magnfitude desribed in the simulation framewrok above, and ask,for each gene, in how many tissues is the effect a certain  proportion, suppose 50\%, of the maximum effect? Again, homogenous genes will tend to be featured at the right of the distribution, with the majoirty of their tissues effect sizes similar in magnitude, while meterogneous genes will be featured towards the left, with tissue-specific genes at the extreme left. Again, excluding brain form the anlyais tends to nudge us towards a belief in consistency. 

\textbf{Magnitude Heterogeneity Index}

Attempting to understand which genes tend to behave the most homogenously or heterogneously, we can plot the value used to nomralize each gene, e.g., the 'maxium' effect size across tissue ofr the gene, vs the normalized values. We can see that if a large effect is present, it tends to be in the presence of homogenous effects across the borad, while small normalizing effects tend to be in the presence of effects that are more variable in sign and magnitude. 

\textbf{BIPLOT}


\begin{table}[htbp]
\caption{Heterogeneity Comparison}
\centering
\begin{tabular}{c c c c}
\hline\hline
Metric & Matrix Ash  & MatrixAshNOB  \\ [0.5ex] % inserts table %heading
\hline
Consistent in sign (normalized value $>$ 0): E(Z$\mid$D)$ &0.833&0.880 \\
%E(DifferentSign (no threshold)&0.906 ( 0.87 no brains) &0.597&0.481\\
E(DifferentSignPosteriorMean$\mid$ LFSR$\leq$0.05)&0.198&0.148\\
E(At least 50\% max value) &0.354&0.449\\
%Sign Change &0&0.184&0.179\\[1ex]
\hline
\end{tabular}
\label{table:nonlin}
\end{table}


%In how many genes do there exists effects of a different sign? 
%9,597 (59.7%) vs 7,723 (48.1%)
% What about effects at a given significance threshold?
%3,180 with brain includes (19.8%), 2,377 no brain at an LFSR threshold of 0.05 ( 14.8%)
%Gene.snp.tissue.effects (i.e., b.j.r)
%With brains: Proportion b.j.r norm>0: 83.3 % Without brains: 88%
%Proportion b.j.r norm>50%: 35.4 vs 44.9%

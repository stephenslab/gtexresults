\section{Results}
\subsection{Demonstrating Features of the Method}

To get a sense of the accuracy of our novel approach to estimating multivariate effects, we simulated two types of data. 

In the first set, in which we expect our method to be superior to both univariate methods and methods in which the configuration approach is utilized, we simulate 50,000 gene-snp pairs, with only 400 representing true signal. This represents roughly 500 genes with 100 snps in cis, $80\%$ of which contain one active QTL. Thus naturally, if the gene contains such a QTL, it is the same QTL among all tissues in which the tissue is active. This puts a dual burden on both features of the method: The small number of true associations present in these simulations tests whether the method accurately encourages small observed effects toward zero while preserving the true signal when it exists. Furthermore, the  multivariate nature of these simulated effects when they exist tests the ability of the method to accurately infer patterns of sharing from the dataset. These true effects are thus simulated from the `learned' covariance matrices representing $U_{k}$ 2-9, and thus aim to emulate the patterns of sharing present in real biological data. We compare with univariate `shrinkage' method Ash (Stephens et al, unpublished) as well as the eqtlBMA-lite (Flutre et al, 2013) which uses the singleton and fully consistent (i.e., active in only one tissue, or active with the same effect size in all tissues) configurations to estimate these effects jointly. We call this the `sharing' (S) scenario. 

One might expect that our method would prove superior only in the setting in which true effects are shared among all tissues, and thus fail in the setting of tissue specificity. Thus, building on the situation above, we add a simulation in which $35\%$ of the true effects are active in only one tissue, according to 5 different patterns of tissue specificity. We call this the `tissue-specific' (TS) scenario. 

We see that in terms of both power and accuracy, Matrix Ash is superior in both setting to both univariate methods, and to joint analysis under the setting of orthogonal configurations.
\begin{equation}
RMSE \sqrt(\sum_{jr}(b_{jr}-E(b_{jr}|Data)^2))
\end{equation}
\begin{table}[ht]
\caption{Accuracy Comparison: RMSE}
\centering
\begin{tabular}{c c c c}
\hline\hline
Inference Method & MASH & ASH & eqtlBMA-lite \\ [0.5ex] % inserts table %heading
\hline
RMSE_{S}&0.010&0.030&0.047\\
RMSE_{TS}&0.008& 0.025&0.043 \\%&8649
cor.with.truth_{S}&0.99&0.94&0.84\\
cor.with.truth_{TS}&.99&0.94&0.82\\
\hline
\end{tabular}
\label{table:RMSE}
\caption{\textbf{Accuracy Analysis} Here, we compare the ability of matrix ash (`MASH') to capture the true effect size estimates. We compare with univariate-shrinkage method `ASH' and configuration-specific joint approach `eqtl-BMA-lite'. We report the Root Mean Squared Error (RMSE) and the correlation with the truth.}
\end{table}

To demonstrate the ability of Matrix Ash to powerfully capture these accurately estimated effect sizes, we compare the proportion of true associations called significant at a given significance threshold among the three methods. Indeed, Matrix Ash  proves superior to both methods under each condition (i.e., sharing or tissue-specific). \newline


\textbf{Figure: Power vs Accuracy Figure}
\begin{figure}[h]
\includegraphics[width=5cm]{Figures/PlotsForPaper_files/figure-html/plot1.png}&
\includegraphics[width=5cm]{Figures/PlotsForPaper_files/figure-html/unnamed-chunk-4-1.png}
\end{figure}\newline

In introducing a method to quantify the heterogeneity of effect sizes, we have developed a `heterogeneity index' which attempts to capture the heterogeneity in magnitude among tissues in which the gene-SNP pair is active. For each gene-snp pair $\textbf{j}$, we normalize its vector of effects $\textbf{b_{j}$ across tissues by the effect which has the maximum absolute value; thus for a fully `consistent' gene-snp pair in which all the effects are equal in magnitude, the new vector of normalized effects would consist of all ones, and R=44 tissues would be greater than $50\%$ of the maximum effect. By contrast, for a tissue-specific gene-snp pair, the vast majority of effects would be small fraction of the maximum effect and thus the number of tissues greater than $50\%$ of the maximum effect would be 1 (the effect used to normalize). We can apply this heterogeneity index, here deemed `HI' to the real data, but first wanted to demonstrate the superiority of Matrix Ash in estimating these quantities on simulated data. To quantity the ability of each method to accurately ascertain the heterogeneity, we can compute the heterogeneity index of the real data, and the inferred quantities, and use a modified RMSE:

\begin{equation}
RMSE_{HI} \sqrt(\sum(true_{HI}-estimated_{HI})^2))
\end{equation}

\begin{table}[ht]
\caption{Accuracy Comparison: RMSE}
\centering
\begin{tabular}{c c c c}
\hline\hline
Inference Method & MASH & ASH & eqtlBMA-lite \\ [0.5ex] % inserts table %heading
\hline
HI_{S}&39.38 &40.87 &39.78 \\
HI_{TS}& 39.98& 40.77&39.51\\
\hline
\end{tabular}
\label{table:HETindex}
\end{table}

\subsection{Adaptive Shrinkage: The Multivariate Approach}

To demonstrate the utility of shrinking effect size estimates jointly, we consider the estimated effect sizes vs their observed input summary statistics using our joint (Matrix Ash) and comparing to a univariate shrinkage method (Ash). On simulated data, we can also then plot the estimated effect sizes against the true values,  again comparing among methods. Here, we show the results under the setting of tissue specificity, to analyze the behavior of eQTL of each class.\newline

\textbf{Figure: Insert Simulations from TSpecific Conditions here}
\begin{figure}[htbp]
\includegraphics[width=5cm]{Figures/scatterplot_fittedtspec.png}&
\includegraphics[width=5cm]{Figures/scatterplot_fittedtspec_ash.png}\\
\hfill
\includegraphics[width=5cm]{Figures/scatterplot_truthtspec.png}&
\includegraphics[width=5cm]{Figures/scatterplot_TRUTHashtspec.png}
\end{figure}\newline





In both Matrix Ash and univariate methods, values with large standard errors will be shrunk more harshly (Ash Paper, Stephens et al). Comparing estimated Z statistics (i.e., $E(Z_{jr}|Data$) vs the observed `raw' input values (i.e., $\hat{b_{jr}}$ allows us to understand the behavior of multivariate vs univariate methods once the standard error has been considered. In this simulated data, where there is an abundance of small effects, both univariate and multivariate methods tend to shrink small observed values towards prior mean at $\bm{0}$ as their likelihoods will be maximized by component with small $\omega$. Critically, considering observed Z statistics of the same size, Matrix Ash does not shrink all small values are shrunk to the same extent, due to the power of joint analysis to consider the effects across tissues in inferring the final vector of effect sizes. Thus the method is dually `adaptive' by considering the abundance of both effect sizes and shapes in the overall data-set.  Here, acknowledging consistency, small effects in one tissue will be augmented in the presence of larger effects  other tissues, resulting in dramatic power increases. 

Furthermore, when we plot the estimated effects against truth and segregate these effects by class (e.g., active and shared, active and tissue-specific, or null), we see that the correlation among the true and estimated effect $(E(Z|D)$ sizes is much tighter using our multivariate approach. Similarly, truly null effects are shrunk more tightly, due to the fact that in the presence of consistency, small effects across subgroups will lead us to have a high prior belief that an additional small observed effect in that eQTL is also likely to be close to 0. Importantly, tissue-specific QTLs are still captured using our joint approach, demonstrating that if tissue-specific patterns exist in the data, our prior belief will capture this phenomenon and accordingly our posterior estimates will reflect the underlying tissue-specific nature at a given tissue-specific SNP.

Together, these results demonstrate the tremendous power increase of using a multivariate method and the accuracy of estimating patterns of sharing from the data rather than imposing forced configurations which fail to capture the heterogeneity of effect sizes among tissues.


\section{Real Data}

%Now, we consider the results of our analysis, when applied to the GTEX data set. After estimating the covariance matrices from the strongest gene-snp pairs, in an effort to capture the underlying 'true patterns' of sharing in the data and adding the qualitatively specific configurations `bma-lite' configurations, we can infer the relative frequency of each pattern of sharing and corresponding effect sizes from a large sample of 40,000 gene-snp pairs. 

Here, we report the analysis on the top SNP for each of 16,069 genes, where the `top' snp is defined as the SNP with the largest observed univariate Z-statistic in absolute value across tissues. As described above and demonstrated in simulations, in the setting of an abundance of small effects in data set, Matrix Ash tends to shrink small observed values towards the prior mean at $\bm{0}$. It should be noted that this is a result specific to a particular data set, and in that sense `adaptive' - indeed, if small effects were rare and large effects abundant, such shrinkage would not occur. 

But perhaps more importantly, the striking increase in power when compared to univariate methods is noted. There are a total of 44 tissues x 16,069 gene-snp pair associations considered, or 707,036 total tissue-level effect size coefficients. At an $lfsr$ threshold of 0.05, we identify 393,414 significant snp-gene-tissue effects ($b_{jr}$). Using estimates shrunk according to a univariate approach (again, Ash),  we identify only 91,755, meaning that using univariate methods we would be confident our ability to identify the sign in only $13\%$ of cases, while using our joint procedure for estimating effects, we would confidently argue the SNP has a non-zero effect for a gene in a particular tissue over half $(55\%)$ of the time. As described, this tremendous increase in power arises from the fact that in the presence of  a data set possessing consistency, as learned by the hierarchical model, small effects in the presence of a gene containing large effects in alternative tissues will be augmented to reflect such consistency, thus increasing our confidence in its size and direction. While the number of associations capture is slightly greater using the BMA lite approach, we note that the likelihood of the data set under this model is much much worse ($-1298672$ vs $-1267997.5$, see supplementary data `Testing and Training' procedure). Indeed, eQTLBMA would put the vast majority of the prior weight on the fully `consistent' configuration ($\hat{\pi}}$ figure, as SNPs demonstrating activity across all tissues, regardless of how heterogeneous among subgroups, are forced into this configuration. Simulations above demonstrate the lack of accuracy arising from such an approach. \newline


\textbf{Figure: Real Data Scatterplot}
\begin{figure}[htbp]
\includegraphics[width=5cm]{Figures/comparebeta.png}&
\includegraphics[width=5cm]{Figures/comparez.png}\\
\end{figure}\newline

\begin{table}[ht]
\caption{Power Comparison}
\centering
\begin{tabular}{c c c c}
\hline\hline
Metric & LFSR_{Matrix Ash} & LFSR_{ASH}&eQTL-BMALite \\ [0.5ex] % inserts table %heading
\hline
Significant $\bf{b}_{jr}$ $\leq$ 0.05%&202087
&393414 & 91,755&401552\\
%$\bf{b}_{jr}$ significant in other not in MASH %&8649
%&NA&1447 \\
%$\bf{b}_{jr}$ significant in MASH not in other %&199976
%&NA&303106 \\[1ex]
\hline
\end{tabular}
\label{table:power}\newline
\caption{\textbf{Power} Restricting our analysis to thresholding by local false sign rates, we can quantify the number of associations we identify at a given local false sign rate threshold using the original summary statistics and posterior means computed using multivariate Matrix Ash and Univariate Ash. We can see that Matrix Ash calls nearly twice and 4 times as many associations significant when compared to univariate approach, and is comparable to less-accurate joint approach}
\end{table}\newline

\textbf{Figure: $\hat{\pi}$ barplot in MASH vs BMAlite}
\newline
\begin{figure}[htbp]
\includegraphics[width=10cm]{Figures/PlotsForPaper_files/figure-html/unnamed-chunk-9-1.png}
\end{figure}\newline


To further contrast our approach with existing joint methods on this data-set, consider a two-tissue example, in which a configuration type approach recognizes only patterns constrained to lie along the x and y axis or along the x-y line. Matrix ash allows for patterns which show consistently larger effects in one tissue over another, with varying amounts of correlation among tissues. In these example from real data, we can see that while eQTL  the green, blue and yellow class appear consistently active in both tissues plotted (Brain and Muscle), the blue and yellow effects have consistently larger effects in Skeletal Muscle than brain, while SNPs of the green class show the reverse. Similarly, eQTL of the yellow close show only loose correlation among the pair of tissues, while eQTL of the green class show strong prediction of activity in one from activity in the other.  Similarly, we can see that eQTL of the pink class tend to show tissue-specificity in Whole Blood relative to testis, while eQTL of the black class show tissue specificity in testis relative to whole blood.\newline


\textbf{Figure: Colored Scatterplots of 2 tissues, contrasting effects between tissues, color coded by responsibility}
\newline

\begin{figure}[htbp]
\includegraphics[width=5cm]{Figures/PlotsForPaper_files/figure-html/unnamed-chunk-8-2.png}&
\includegraphics[width=5cm]{Figures/PlotsForPaper_files/figure-html/unnamed-chunk-8-3.png}\\
\hfill
\includegraphics[width=5cm]{Figures/PlotsForPaper_files/figure-html/unnamed-chunk-8-4.png}&
\includegraphics[width=5cm]{Figures/PlotsForPaper_files/figure-html/unnamed-chunk-8-5.png}
\end{figure}\newline


\subsection{A qualitative description of heterogeneity in the GTEX data}
\begin{figure}[htbp]
\includegraphics[width=10cm]{Figures/gtexresultcompiledheatmaps.png}
\end{figure}\newline

 
Indeed, from the prior weight assigned to the `learned matrices' above coupled with the simulation results in the previous sections, we can see that Matrix Ash is able to accurately parse shared configurations, thus resolving the relationship among tissues in which the QTL is active. \textit{See Supplement for Heatmap of all 8 learned matrices}. 

For instance, learned matrix $U_{k}$ = 3 captures gene-snp pairs with large, correlated effects in brain, and is the most prevalent pattern of sharing in the larger data set, as reflected by it's prior weight summed across effect size (see pi barplot). Matrix $U_{k}$ = 2 captures SNPs with small effects in brain and larger effects in thyroid and transformed cell-types (e.g., fibroblasts, lymphocytes). Several of the lower rank matrices whose patterns receive high prior weighting (e.g., $U_{k} = 4,5,8 and 9$) show somewhat tissue specific (i.e., high prior variance in only one tissue-type) effects in testes and whole blood, consistent with our conclusions that whole blood and testes indeed demonstrate an abundance of tissue-specific gene-snp pairs. Here we examine several example gene-snp pairs with a high posterior probability of arising from the covariance patterns captured by our model. We deem this posterior probability of arising from a particular pattern as a high `loading' or `responsibility.' 


\subsection{Examples of strong loading on UK3, UK9 and eQTL-BMA lite}

\newline
\textbf{Figure: High loading on UK3: Captures correlation in sign, quantitative heterogeneity in magnitude along diagonal emphasizes the utility of continuous approach}
\newline
\begin{figure}[htbp]
\includegraphics[width=10cm]{Figures/PlotsForPaper_files/figure-html/unnamed-chunk-12-1.png}\\
\includegraphics[width=10cm]{Figures/PlotsForPaper_files/figure-html/unnamed-chunk-12-2.png}
\end{figure}\newline


In this particular example, strong, shared effects in brain tissues match an underlying pattern of shared effects present in the larger data set and thus allows this gene-snp pair to find its true match. Brain effect sizes thus borrow strength from one another, and accordingly, the posterior estimates tend to nudge the brains towards a consistent, shared effect. Similarly, an overall tendency towards consistency in sign in the larger data set, as captured by the hierarchical model and reflected in the positive correlation in sign among all tissues, tends to `flip' erratic off directions towards the prevailing positive direction. Heterogeneity in magnitude among the other tissues is reflected in the variety of banding intensity along the diagonal.\newline


\textbf{Figure: UK 9: Quantitative specificity in magnitude in Testes/Whole Blood}
\newline
\begin{figure}[htbp]
\includegraphics[width=10cm]{Figures/PlotsForPaper_files/figure-html/unnamed-chunk-13-1.png}\\
\includegraphics[width=10cm]{Figures/PlotsForPaper_files/figure-html/unnamed-chunk-13-2.png}
\end{figure}\newline


In this example, though the particular pattern featured ($U_{k}=9$) captures correlation in sign among all tissues, significant quantitative heterogeneity is again reflected in the intensity of the banding along the diagonal, in this dramatically dichotomous between testes/whole blood and all other tissues. Here, we introduce the idea of quantitative specificity - e.g., that a SNP can be modestly `active' in all tissues though to dramatically different degrees. here, though this matrix was learned (and not forced, as in eQTLBMA-lite) from the data, the pattern of quantitative tissue specificity in testes and whole blood is evident. Again, erratic, off-directions are flipped in sign. We refer to this as quantitative specificity, because the effects are quantitatively unique to particular tissues - e.g., significantly larger in magnitude in testis than all other tissues - and yet considered non-zero in all tissues. This is in contrast to qualitative specificity, described below, in which we would conclude that the QTL is active in only one tissue. \newline



\textbf{Figure: Qualitatitive specificity in Testis Example: High loading on eqtlbmalite config mat}
\newline
\begin{figure}[htbp]
\includegraphics[width=10cm]{Figures/PlotsForPaper_files/figure-html/unnamed-chunk-14-1.png}\\
\end{figure}\newline

Lastly, the inclusion of the eQTLBMA lite configurations (in which the SNP has a non-zero effect in only one tissue) coupled with the learned patterns of tissue specificity evident in matrices $U_{k}: 5-9$ serve to allow the preservation of qualitatively specific effects. Here, we show a gene-snp pair demonstrating high loading on one of the eqtlbma-lite configuration matrices - indeed, we reject the significance of the effect size estimates in all tissues but testes, a pattern consistent with the presence of tissue-specificity described below. Together, these results cement the resolution afforded by methods which can distinguish among tissues in which a QTL is called active, beyond reducing genetic effects to binary `on' or `off' conclusions.


\subsection{Tissue Specificity}

One of the criticisms of a joint approach might be its loss of tissue-specificity. That is, by considering effects across subgroups in estimating the effect size, one might lose sight of tissue-specific activity when it exists. Here, we demonstrate our ability to recognize such specificity both quantitatively, as described above through learned patterns of sharing which specify consistently larger effects in one tissue over others, and qualitatively through forced prior effect size mass on 0.  For each tissue, we can ask how many gene-snp pairs meet a given significance threshold in that tissue alone.Furthermore, tissue specific eQTL demonstrate the smoothing feature of this joint shrinkage approach: for gene SNP pairs which demonstrate strong effects in only one tissue, the weaker erratic tissue are shrunk towards the prior mean at $\bm{0}$, resulting in a tissue specific smoothing.\newline


\textbf{Figure: Tissue Specific Smoothing Plot and NumberTissueSpecific}
\newline
\begin{figure}[htbp]
\includegraphics[width=10cm]{Figures/PlotsForPaper_files/figure-html/unnamed-chunk-15-1.png}\\
\end{figure}\newline




\subsection{Quantifying Heterogeneity}


Armed with a vector of effect size estimates across 44 tissues, $\bm{b_{j}}$, we can move beyond asking in how many tissues is a given gene-snp pair significant, and ask about the relationship in effect size and direction among tissues in which the gene-snp pair is active.

\textbf{Figure: Number of QTL per Tissue Plot}
\newline
\newline
\begin{figure}[htbp]
\includegraphics[width=5cm]{Figures/PlotsForPaper_files/figure-html/unnamed-chunk-16-1.png}\\
\end{figure}\newline

 From a biological standpoint, we might produce that effects of a different sign are rare. Similar to the heterogeneity index described in the simulation framework above which attempted to describe heterogeneity in magnitude, we can plot the number of tissues in which the sign is differ than the effect with maximum absolute value. Considering this results with and without the inclusion of the brain tissues, which appear to behave as a strongly correlated group, we observe several phenomenon. The majority of gene-snp pairs are consistent in sign (indeed, only about $20\%$ of genes show two significant effects of a different sign when including brain, and even fewer $(14.8\%)$ when excluding brains) and removing brains from our analysis tends to push us towards consistency, suggesting that brain appears to behave as a large tissue-specific entity. After normalizing each gene-snp effect size coefficient $b_{jr}$ by the effect size with the maximum value for the gene, we can also ask what proportion of these are positive. We come to similar conclusions with $83\%$ and $87\%$ with and without brain demonstrating positive normalized effects, respectively. \newline


\textbf{Figure: Sign Heterogeneity Distribution Figure}
\newline
\begin{figure}[htbp]
\includegraphics[width=10cm]{Figures/PlotsForPaper_files/figure-html/unnamed-chunk-20-1.png}\\
\end{figure}\newline

Furthermore, we can now quantify the heterogeneity index in magnitude described in the simulation framework above, and ask, for each gene, in how many tissues is the effect greater than equal to a significant fraction, here $50\%$ of the maximum effect. Again, homogenous genes will tend to be featured towards the right of the distribution with maximal value at 44. Intuitively, the majority of their effects across tissues are similar in magnitude, while heterogeneous genes will be featured towards the left of the distribution, with tissue-specific genes posessing a heterogeneity index of 1. Again, excluding brain form the analysis tends to nudge us towards a belief in consistency. 

Taken together, these results suggest the presence of consistency in sign in our data set, and a bimodal distribution of heterogeneity in magnitude.

\begin{table}[htbp]
\caption{Heterogeneity Comparison}
\centering
\begin{tabular}{c c c c}
\hline\hline
Data & All Tissues  & No Brains  \\ [0.5ex] % inserts table %heading
\hline
Consistent in Sign $E(b_{jrnorm}|D)$ $>$ 0) &0.833&0.880 \\
%E(DifferentSign (no threshold)&0.906 ( 0.87 no brains) &0.597&0.481\\
E(Consistent SignPosteriorMean$\mid$ LFSR$\leq$0.05)&0.802&0.852\\
E(At least 50\% max value) &0.354&0.449\\
%Sign Change &0&0.184&0.179\\[1ex]
\hline
\end{tabular}
\label{table:nonlin}
\caption{\textbf{Heterogeneity Analysis} After normalzizing each gene-snp-effect size coefficient by the effect size with maximal value at that gene, we can ask how many of these gene-snp effect coefficients are positive. Similarly, at a given significance threshold, we can ask how many gene-pairs contain effects of different signs across tissues. At an arbitrary LFSR threshold of 0.05 for instance, we note that $80\%$ of genes are homogenous in sign when all tissues are considered. Excluding brains from our analysis, this rises to $85\%$. To evaluate consistency in magnitude, we can ask how many gene-snp-tissue effects are greater than $50\%$ of the maximal effect across tissues for the pair. Again, we see that excluding brains from our analysis tends to push this towards consistency.}
\end{table} \newline


\textbf{Figure: Magnitude Heterogeneity Index Distribution}
\newline
\begin{figure}[htbp]
\includegraphics[width=10cm]{Figures/PlotsForPaper_files/figure-html/unnamed-chunk-23-1.png}\\
\end{figure}\newline


Attempting to understand which genes tend to behave the most homogeneously or heterogeneously, we can plot the value used to normalize each gene, e.g., the `maximum' effect size across tissue of the gene, against the normalized values. We can see that if a large effect is present, it tends to be in the presence of homogenous effects across the board, while small normalizing effects tend to be in the presence of effects that are more variable in sign and magnitude. Furthermore, aggregating the gene-snp pairs at a given heterogeneity index and classifying them by the effect used to normalize (e.g., the `max effect') we can see that gene-snp pairs with greater Heterogeneity indices tend to have larger effects. \newline


\textbf{Figure: Insert Plot of Normalized Effect vs Max value (e.g., biplot)}
\newline
\begin{figure}[htbp]
\includegraphics[width=5cm]{Figures/normstuffeb_nobrain.png}&
\includegraphics[width=5cm]{Figures/normstuffeb_alltissues.png}&
\end{figure}\newline

\textbf{Figure: Insert Median Max Effect by HI Index}
\newline
\begin{figure}[htbp]
\includegraphics[width=5cm]{Figures/PlotsForPaper_files/figure-html/unnamed-chunk-24-1.png}\\
\end{figure}\newline

\textbf{Figure: QTL chart by HI index: homogenous or heterogenous}




%In how many genes do there exists effects of a different sign? 
%9,597 (59.7%) vs 7,723 (48.1%)
% What about effects at a given significance threshold?
%3,180 with brain includes (19.8%), 2,377 no brain at an LFSR threshold of 0.05 ( 14.8%)
%Gene.snp.tissue.effects (i.e., b.j.r)
%With brains: Proportion b.j.r norm>0: 83.3 % Without brains: 88%
%Proportion b.j.r norm>50%: 35.4 vs 44.9%
